\documentclass[12pt,t]{beamer}
\usetheme{default}
\beamertemplatenavigationsymbolsempty
\hypersetup{pdfpagemode=UseNone} % don't show bookmarks on initial view
\usefonttheme{professionalfonts}
\usefonttheme{serif}
\usepackage{fontspec}
\setmainfont{FreeSerif}
\setmonofont{FreeMono}
\usepackage{url}
\usepackage[english]{babel}
\usepackage{verbatim}
%\usepackage{times}
\usepackage{graphicx}
\usepackage[utf8]{inputenc}
%\usepackage[T1]{fontenc}
\usepackage{listings}
\usepackage{mathtools}
\usepackage{color}
\usepackage{minted}
\usemintedstyle{vim}
\usepackage{tikz}
\usepackage[underline=false]{pgf-umlsd}
\usetikzlibrary{positioning, fit, calc, shapes, arrows, shadows}
\usepgflibrary{arrows}
\definecolor{lightgray}{gray}{0.70}

\definecolor{fore}{RGB}{249,242,215}
\definecolor{back}{RGB}{51,51,51}
\definecolor{title}{RGB}{200,0,90}
\setbeamercolor{titlelike}{fg=title}
\setbeamercolor{normal text}{fg=fore,bg=back}

\definecolor{foreground}{RGB}{255,255,255}
\definecolor{background}{RGB}{24,24,24}
\definecolor{title}{RGB}{107,174,214}
\definecolor{gray}{RGB}{155,155,155}
\definecolor{subtitle}{RGB}{102,255,204}
\definecolor{hilight}{RGB}{102,255,204}
\definecolor{vhilight}{RGB}{255,111,207}
\setbeamercolor{titlelike}{fg=title}
\setbeamercolor{subtitle}{fg=subtitle}
\setbeamercolor{institute}{fg=gray}
\setbeamercolor{normal text}{fg=foreground,bg=background}
\setbeamercolor{item}{fg=foreground} % color of bullets
\setbeamercolor{subitem}{fg=lightgray}
\setbeamercolor{itemize/enumerate subbody}{fg=lightgray}
\setbeamertemplate{itemize subitem}{{\textendash}}
\setbeamerfont{itemize/enumerate subbody}{size=\footnotesize}
\setbeamerfont{itemize/enumerate subitem}{size=\footnotesize}
\newcommand{\funcname}[1]{
	{\color{yellow!30} #1}
}
\newcommand{\cipher}[1]{
	{\color{blue!30} #1}
}
\setbeamertemplate{footline}[text line]{%
  \parbox{\linewidth}{\vspace*{-8pt}
  \hfill
  %\insertshortauthor
  \hfill
  \insertframenumber\ of \inserttotalframenumber}}
\setbeamertemplate{navigation symbols}{}

\title{abc}
\author{Vsevolod Stakhov \\ \url{vsevolod@FreeBSD.org}}
\institute{\includegraphics[scale=0.5]{logo.pdf}}

\newcommand{\bloodymess}[7][0]{
  \stepcounter{seqlevel}
  \path
  (#2)+(0,-\theseqlevel*\unitfactor-0.7*\unitfactor) node (mess from) {};
  \addtocounter{seqlevel}{#1}
  \path
  (#4)+(0,-\theseqlevel*\unitfactor-0.7*\unitfactor) node (mess to) {};
  \draw[->,>=angle 60] (mess from) -- (mess to) node[midway, above]
  {#3};

  \if R#5
    \node (#3 from) at (mess from) {\llap{#6~}};
    \node (#3 to) at (mess to) {\rlap{~#7}};
  \else\if L#5
         \node (#3 from) at (mess from) {\rlap{~#6}};
         \node (#3 to) at (mess to) {\llap{#7~}};
       \else
         \node (#3 from) at (mess from) {#6};
         \node (#3 to) at (mess to) {#7};
       \fi
  \fi
}
\begin{document}

\begin{frame}[plain]
  \titlepage
\end{frame}

\begin{frame}
\frametitle{How expensive is encryption nowadays}
\begin{itemize}
  \item<1-> New hardware:
  \begin{itemize}
    \item specialized encryption instructions (AES-NI)
    \item vectorized operations (SSE, AVX, AVX2, AVX512)
  \end{itemize}
  \item<2-> New algorithms:
  \begin{itemize}
    \item optimized chaining mode (e.g. CTR instead of CBC)
    \item optimized algorithms (from \cipher{3DES} to \cipher{ChaCha20})
  \end{itemize}
  \item<3-> New protocols
\end{itemize}
\end{frame}

\begin{frame}
\frametitle{How expensive is encryption nowadays}
\framesubtitle{Hardware performance}
2011: Westmere (SSE4, AES-NI):
\begin{figure}[H]
\includegraphics[height=0.6\textheight]{perf-e7.pdf}
\caption{XeonE7, 2.1 GHz, 8 CPU cores}
\end{figure}
\end{frame}

\begin{frame}
\frametitle{How expensive is encryption nowadays}
\framesubtitle{Hardware performance}
2012: Sandy Bridge (AVX, AES-NI):
\begin{figure}[H]
\includegraphics[height=0.6\textheight]{perf-e3.pdf}
\caption{Xeon E3, 3.4 GHz, 8 CPU cores}
\end{figure}
\end{frame}

\begin{frame}
\frametitle{How expensive is encryption nowadays}
\framesubtitle{Hardware performance}
2013: Haswell (AVX2, AES-NI):
\begin{figure}[H]
\includegraphics[height=0.6\textheight]{perf-i7.pdf}
\caption{Core-i7 4770, 3.5 GHz, 8 CPU cores}
\end{figure}
\end{frame}

\begin{frame}
\frametitle{How expensive is encryption nowadays}
\framesubtitle{Hardware performance}
Pre-historic ages: Core2 quad (SSE3):
\begin{figure}[H]
\includegraphics[height=0.6\textheight]{perf-c2.pdf}
\caption{Core2-quad, 1.5 GHz, 4 CPU cores}
\end{figure}
\end{frame}

\begin{frame}
\frametitle{How expensive is encryption nowadays}
\framesubtitle{Algorithm performance} 
\cipher{3DES}:
	\begin{itemize}
	\item very small block size (64 bits)
	\item need to rekey after 32 GB data
	\item need some additional MAC algorithm
	\item terribly slow
	\end{itemize}
\end{frame}

\begin{frame}
\frametitle{How expensive is encryption nowadays}
\framesubtitle{Algorithm performance} 
\cipher{ChaCha20}:
%XXX: picture here
\end{frame}

\begin{frame}[fragile]
\frametitle{How expensive is encryption nowadays}
\framesubtitle{Algorithm performance} 
\cipher{ChaCha20} round:
\begin{tiny}
\begin{minted}{c}
void qr(a,b,c,d) {
    a += b; d ^= a; d <<<= 16;
    c += d; b ^= c; b <<<= 12;
    a += b; d ^= a; d <<<= 8;
    c += d; b ^= c; b <<<= 7;
}

for (i = chacha_rounds;i > 0;i -= 2) {
    qr(x0, x4, x8,x12)
    qr(x1, x5, x9,x13)
    qr(x2, x6,x10,x14)
    qr(x3, x7,x11,x15)
    qr(x0, x5,x10,x15)
    qr(x1, x6,x11,x12)
    qr(x2, x7, x8,x13)
    qr(x3, x4, x9,x14)
}
\end{minted}
\end{tiny}
\begin{itemize}
\item<2-> Each round modifies the whole block
\item<2-> Each quarter round operation is independent from others in the round
\item<2-> Efficient diagonal optimizations for SSE/AVX/AVX512
\item<2-> Need even number of rounds (20 or 12 typically)
\end{itemize}
\end{frame}

\begin{frame}
\frametitle{How expensive is encryption nowadays}
\framesubtitle{Algorithm performance} 
Advantages of \cipher{ChaCha20}:
\begin{itemize}
\item clear and simple design
\item 512 bits of block size (up to $2^{70}$ bytes before rekeying)
\item fit very well for vectorized operations (and especially for AVX512)
\end{itemize}
Current usage of \cipher{ChaCha20-Poly1305}:
\begin{itemize}
\item openssh
\item libressl and boringssl
\item libottery fast pseudo-random generator
\item OpenBSD \funcname{arc4random}
\item Chrome browser
\item proposed IETF standard for TLS and IPSEC
\end{itemize}
\end{frame}

\begin{frame}[fragile]
\frametitle{How expensive is encryption nowadays}
Protocols performance:
\begin{figure}[H]
\centering
\begin{sequencediagram}
\newthread{c}{Client}
\newinst[3]{s}{Server}
\bloodymess{c}{}{s}{R}{ClientHello}{}
\begin{sdblock}[gray!20]{Certificates}{}
\bloodymess{s}{CertificatesChain}{c}{L}{}{}
\end{sdblock}
\bloodymess{c}{ClientKeyExchange}{s}{R}{CertificateVerify}{}
\bloodymess{s}{}{c}{L}{ChangeCipher}{}
\end{sequencediagram}
\caption{TLS connection establishment}
\end{figure}
\end{frame}

\begin{frame}
\frametitle{Performance: summary}
\begin{itemize}
\item<1-> Experimental support of \cipher{AES-GCM} in OpenBSD and FreeBSD (by John-Mark 
Gurney), 
problems with FPU locking
\item<2-> Support of \cipher{ChaCha20-Poly1305} in libressl, boringssl and libgrypt (and 
gnutls)
\item<3-> Deprecating of \cipher{RC4} in OpenBSD (proposed in FreeBSD as well)
\end{itemize}
\end{frame}

\begin{frame}
\frametitle{Building secure systems}
General problems when building secure systems:
\begin{itemize}
\item<1-> Curse of backward compatibility
\item<2-> Complex and inconsistent API (OpenSSL)
\item<3-> Bad default settings
\item<3-> Permit everything by default
\end{itemize}
\end{frame}

\begin{frame}
\frametitle{Building secure systems}
\framesubtitle{Backward compatibility}
\begin{itemize}
\item<1-> Protocols design flaws
\item<2-> Poor algorithmic choices:
	\begin{itemize}
	\item Reduce security
	\item Remove important properties, e.g. forward secrecy property
	\item Can be very slow (\cipher{3DES}) and hence lead to computational DoS
	\end{itemize}
\end{itemize}
\end{frame}

\begin{frame}
\frametitle{Building secure systems}
\framesubtitle{Backward compatibility}
Practical example - LibreSSL replacement of OpenSSL
\begin{itemize}
\item Legacy API - \cipher{DES} support in OpenLDAP
\item Invalid random number generators - \funcname{RAND\_egd} which is valid merely for Linux (Python, Curl, Wget and many other ports)
\item SSL3 and even SSL2 support (Curl)
\item Unnecessary engine functions (many ports, including apache) 
\end{itemize}

Practically \textbf{all} the issues listed are also valid for the upcoming 1.0.2 branch.
\end{frame}

\begin{frame}
\frametitle{Building secure systems}
\framesubtitle{Design flaws example: TLS}
Encrypt and MAC choice.
%XXX demonstrate
\begin{itemize}
\item Sensitive to side attacks (PaddingOracle, LuckyThirteen)
\item Inefficient computation
\item No integrity on ciphertext
\item No protection against the attacks to a cipher itself
\end{itemize}
Proposed: encrypt-then-mac extension in RFC7366.
\end{frame}

\begin{frame}
\frametitle{Building secure systems}
\framesubtitle{Design flaws example: TLS}
CBC mode.
\begin{itemize}
\item Assumes that cipher is secure for all data even if it is controlled by an attacker
\item Decryption differs from encryption
\item Cannot be computed in parallel
\item Needs careful padding (Poodle attack)
\end{itemize}
Proposed: use ciphers in counter mode only (not compatible with old browsers)
\end{frame}

\begin{frame}
\frametitle{Building secure systems}
\framesubtitle{TLS compatibility in the real life}
\begin{figure}[H]
%\includegraphics[width=\textwidth]{compat.png}
\caption{`Secure' TLS compatibility}
\end{figure}
\end{frame}

\begin{frame}
\frametitle{Building secure systems}
\framesubtitle{API design flaws}
\begin{itemize}
\item<1-> all software contain mistakes
\item<2-> complicated API increases chance of mistakes some of them are security 
vulnerabilities
\item<3-> inconsistent API provokes misusage
\item<4-> always prefer simple and widely used libraries
\end{itemize}
\end{frame}

\begin{frame}[fragile]
\frametitle{Building secure systems}
\framesubtitle{API design flaws: OpenSSL}
Certificates verification.
\begin{itemize}
\item Terribly complicated - just look at the documentation of 
\funcname{SSL\_set\_verify} and auxiliary function \funcname{SSL\_set\_ex\_data}:
\begin{tiny}
\begin{minted}{c}
ssl = X509_STORE_CTX_get_ex_data(ctx, SSL_get_ex_data_X509_STORE_CTX_idx());
mydata = SSL_get_ex_data(ssl, mydata_index);
\end{minted}
\end{tiny}
\item You need to check certificate CN manually (not even covered in the example in 
the manual page)
\item You need manually check all extensions such as SNI or ALPN (and many openssl users 
fail to do it correctly)
\end{itemize}
\end{frame}

\begin{frame}[fragile]
\frametitle{Building secure systems}
\framesubtitle{API design flaws: OpenSSL}
Macro based API.
\begin{itemize}
\item Inconsistent: 
\begin{enumerate}
\item Many ways to do the same thing, for example \textbf{EVP} and legacy and obsoleted 
interfaces: \funcname{EVP\_PKEY\_encrypt} and \funcname{RSA\_public\_encrypt}.
\item Confusing names \funcname{PEM\_read\_RSAPublicKey}, 
\funcname{PEM\_read\_RSA\_PUBKEY} 
and \funcname{PEM\_read\_PUBKEY}
\end{enumerate}
\item Dangerous pointers API:
\begin{tiny}
\begin{minted}{c}
unsigned char *sk, *p;
size_t sklen;

sklen = i2d_ECPrivateKey(ec_key, NULL);
sk = malloc(sklen);
p = sk;
i2d_ECPrivateKey(ec_key, &p);
/* p is now at the end of sk */
\end{minted}
\end{tiny}
\end{itemize}
\end{frame}

\begin{frame}[fragile]
\frametitle{Building secure systems}
\framesubtitle{API design flaws: some practical advices}
Higher level libraries:
\begin{itemize}
\item libtls from OpenBSD (formely ressl):
\begin{tiny}
\begin{minted}{c}
#include <tls.h>

struct tls *tls_client(void);
int tls_connect(struct tls *ctx, const char *host, const char *port);
int tls_read(struct tls *ctx, void *buf, size_t buflen, size_t *outlen);
int tls_write(struct tls *ctx, const void *buf, size_t buflen, size_t *outlen);
int tls_close(struct tls *ctx);
\end{minted}
\end{tiny}
\item libsodium (not for TLS)
\begin{tiny}
\begin{minted}{c}

\end{minted}
\end{tiny}
\end{itemize}
Be very careful about trust relationships.
\end{frame}

\end{document}